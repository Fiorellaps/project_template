\section{Conclusiones}

Tras contemplar y discutir todos los resultados obtenidos durante el desarrollo de nuestro estudio cabe mencionar la alta conectividad y \textbf{mantenimiento de coherencia} entre las distintas fases del proceso. Se puede apreciar como los módulos más importantes de nuestra red han sido los que finalmente han permitido descubrir las funciones moleculares y procesos biológicos más trascendentales respecto a la interacción del virus en nuestro organismo. Además, estas funciones fueron claramente confirmadas en apartados posteriores cuando el estudio llevado acabo sobre DrugBank nos proporcionaba una serie de medicamentos cuyos objetivos se centraban en tratar muchas de las consecuencias que estas funciones podían acarrear sobre nuestra salud. Es decir, los descubrimientos llevados a cabo durante este trabajo han mantenido en todo momento una dirección muy definida que nos ha permitido realizar afirmaciones sólidas y cómodamente contrastadas.

Además, la conclusión más importante que cabe mencionar de este trabajo es sin duda la gran cantidad de información útil que nos puede ofrecer el estudio de un interactoma concreto cuando ha sido correctamente acotado y seleccionado bajo condiciones coherentes. En nuestro caso, estudiar la porción del \textbf{interactoma humano} asociado al virus SARS-CoV-2, nos ha permitido conocer no sólo las funciones de las proteínas humanas, sino que a través de dicha información ser capaces de descubrir las posibles interacciones del virus con nuestro organismo, las posibles consecuencias clínicas de dicha interacción, características del propio virus y por último qué medicamentos pueden llegar a ser más efectivos para luchar contra el mismo. Es decir, la red estudiada en este trabajo ha conseguido actuar como una especie de espejo sobre el que hemos podido analizar el reflejo de las partes más esenciales de nuestro virus de interés.

Es evidente que cuantos más trozos de este espejo metafórico tengamos, mayor será la claridad y la precisión con la que podamos analizar el virus y por tanto mayor será el conocimiento que tengamos sobre el mismo. Con esto, queremos señalar la gran importancia que tiene seguir descubriendo nuevos elementos y conexiones del interactoma humano, ya que actualmente se afirma que nuestro conocimiento sobre el mismo es aproximadamente de un 10\%. Ampliar esta gran red o incluso llegar a completarla nos permitiría sin lugar a dudas obtener una mayor cantidad de información con la que poder realizar análisis más profundos y efectivos. Esto también se extendería a estudios más concretos como el nuestro, ya que poseer el interactoma completo asociado al SARS-CoV-2 nos abriría las puertas a nuevos descubrimientos incluidos aquellos que nos permitan eliminar su actual amenaza.
