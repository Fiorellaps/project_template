\section{Introducción}

A finales de 2019 se unió a la familia de los coronavirus el denominado síndrome respiratorio agudo severo coronavirus 2 (SARS-CoV2), cuya infección da lugar al conocido coronavirus 2019 o COVID-19. Este virus ha provocado una pandemia y crisis mundial, dejando por delante un total de más de 2 millones de muertes y casi 96 millones de infectados en todo el mundo. Por ello, desde los equipos de investigación, se está trabajando inmensurablemente por identificar los componentes del mismo, así como sus genes, sus proteínas y las formas en las que estas interaccionan con el ser humano. 

A día de hoy se sabe que el SARS-CoV-2 está formado por 29 proteínas que se asocian con las células humanas dando lugar a múltiples de síntomas que en su máxima expresión pueden llevar a la muerte del individuo con relativa facilidad.

Por otra parte, debemos tener tener en cuenta que el interactoma define el conjunto de interacciones moleculares que tiene lugar en el interior de una célula. El estudio de estas redes biológicas es realmente relevante en diversas áreas de investigación, ya que conocer cómo interaccionan las proteínas entre sí (PPI) permite el descubrimiento de patrones, la elaboración de fármacos o la explicación de procesos biológicos, entre otros.

Es por ello, por lo que en este proyecto pretendemos modelar y estudiar la red de interacción de las proteínas de este virus con las proteínas humanas, así como examinar brevemente si existe algún conjunto de fármacos con targets en las proteínas de la red, observar si la combinación de medicamentos afecta a las enzimas virales, determinar las funciones celulares a las que afecta, etc.
 
Finalmente, tanto los resultados como la presente memoria,  serán proporcionados en el repositorio de github de la asignatura, accesible a todos los miembros del grupo.
