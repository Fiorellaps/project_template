\section{Discusión}

En este estudio hemos partido de la red de interacciones de proteínas humanas con el SARS-cov2 obtenida en \textbf{\textit{‘A SARS-CoV-2 protein interaction map reveals targets for drug repurposing’}} y la hemos \textbf{ampliado hasta obtener 425 interacciones}. Tras eliminar los nodos no conectados, hemos obtenido una componente conexa de 395 nodos. A partir de este grafo hemos aplicado diversas herramientas con el objetivo de obtener más información acerca de su naturaleza, como se ha comentado en el apartado de resultados.


Mediante un \textbf{enriquecimiento funcional} de las 4 comunidades más centrales de nuestra red hemos podido conocer los principales procesos biológicos, funciones moleculares y componentes celulares a los que se asocian las proteínas estudiadas. Tal y como se comenta en  \cite{Gysi2020NetworkCOVID-19}, diversos procesos se relacionan con la actividad cardiovascular del organismo, llegando a provocar desordenes en la presión arterial o en el sistema renina-angiotensina-aldosterona. Esto también lo hemos podido comprobar al estudiar el clúster 84, pues provoca una regulación negativa del óxido nítrico, un compuesto que contribuye a la regulación de la presión arterial.


Otra función que hemos podido descubrir es la participación en la síntesis de lipoproteínas formando los complejos GPI, que permiten el ataque a los procesos de señalización inmunológica. De ahí podemos deducir la efectividad de este virus que es capaz de provocar una inmunodeficiencia severa. Además, la forma en que el virus se propaga en el organismo es particular del SARS-CoV-2.


Finalmente, gracias a la \textbf{búsqueda en Drugbank} no solo hemos podido analizar los medicamentos que se dirigen a las proteínas de nuestra red, sino también nos permite interpretar el mecanismo de acción de los mismos. Por ejemplo, la pravastina podría disminuir las probabilidades de mortalidad del paciente pues es un fármaco que previene el ataque del virus sobre el corazón, mientras que la teofilina podría ayudar a combatir la grave neumonía que provoca el virus en cuestión. 


Además, hemos comprobado cómo los anestésicos volátiles pueden ser una nueva alternativa para la sedación e intubación de pacientes gravemente afectados por el Covid-19 y que efectos puede tener sobre la probabilidad de muerte de los mismos.

